\documentclass[aspectratio=1610,xcolor={usenames,dvipsnames}]{beamer}

\usepackage[utf8]{inputenc}
\usepackage{default}
\usepackage{hyperref}
\usepackage{listings}
\usepackage{xcolor}
\usepackage{lstlanggherkin}
\usepackage{etoolbox}

\definecolor{basrblue}{RGB}{41, 74, 246}
\definecolor{basrorange}{RGB}{242, 42, 42}

\usetheme{basr}
\usecolortheme{basr}

\hypersetup{pdfstartview={Fit}}

\title[Testing] % (optional, only for long titles)
{Coding by Behavior Driven Design}
\subtitle{Examples of workin' with Gherkin}
\author[Rustenburg]{A.S.~Rustenburg,~ Memorial Sloan Kettering Cancer Center}
\date[JCLAB 2014]{Chodera lab group talk, October 17 2014}
\subject{Computer Science}


\newtoggle{InString}{}% Keep track of if we are within a string
\togglefalse{InString}% Assume not initally in string

\newcommand*{\ColorIfNotInString}[1]{\iftoggle{InString}{#1}{\color{basrorange}#1}}%
\newcommand*{\ProcessQuote}[1]{#1\iftoggle{InString}{\global\togglefalse{InString}}{\global\toggletrue{InString}}}%
\newcommand{\python}{\lstset{literate=%
    %{"}{{{\ProcessQuote{"}}}}1% Disable coloring within double quotes
    %{'}{{{\ProcessQuote{'}}}}1% Disable coloring within single quote
    {0}{{{\ColorIfNotInString{0}}}}1
    {1}{{{\ColorIfNotInString{1}}}}1
    {2}{{{\ColorIfNotInString{2}}}}1
    {3}{{{\ColorIfNotInString{3}}}}1
    {4}{{{\ColorIfNotInString{4}}}}1
    {5}{{{\ColorIfNotInString{5}}}}1
    {6}{{{\ColorIfNotInString{6}}}}1
    {7}{{{\ColorIfNotInString{7}}}}1
    {8}{{{\ColorIfNotInString{8}}}}1
    {9}{{{\ColorIfNotInString{9}}}}1,
    language=Python,
    basicstyle=\footnotesize\ttfamily,
    keywordstyle=\color{basrblue}\bfseries,
    keywordstyle=[2]\color{Aquamarine}\ttfamily,
    stringstyle=\color{basrorange}\ttfamily,
    commentstyle=\color{lightgray}\ttfamily,
    morecomments=[l][\color{magenta}]{\#},
    morekeywords=[2]{self},
    showstringspaces=false
}
\renewcommand{\ttdefault}{pcr}	
}

\newcommand{\gherkin}{\lstset{,
    language=[behave]Gherkin,
    basicstyle=\ttfamily,
    keywordstyle=\color{basrblue}\ttfamily,
    keywordstyle=[2]\color{basrorange}\ttfamily,
    stringstyle=\color{basrorange}\ttfamily,
    commentstyle=\color{lightgray}\ttfamily,
    morekeywords=[2]{thing,other_thing},
}
}

\newcommand{\tocframe}{
  \begin{frame}
    \frametitle{Table of Contents}
    \tableofcontents[currentsection,currentsubsection]
  \end{frame}
}


\begin{document}

% Title page
  \begin{frame}
    \titlepage
  \end{frame}

% Table of contents 
  \begin{frame}
    \frametitle{Table of Contents}
    \tableofcontents[]
  \end{frame}

% Reasons for testing code
\section{Why test code?}
\tocframe

\begin{frame}
\frametitle{Why do we write tests for code?}
Writing tests for code can prevent mistakes, and guide development of software.
\begin{exampleblock}{Common reasons for writing tests}
  \begin{itemize} 
    \item<alert@2> Catch bugs introduced during development
    \item<alert@3> Maintain backwards compatibility
    \item<alert@4> Verify output of newly implemented function
    \item<alert@5> Introduction to new developers 
    \item<alert@6> Provide a scaffold for code design
  \end{itemize}
\end{exampleblock}
\end{frame}

    
    
\section{How do we test code in python?}
\subsection{Unit testing}
\tocframe
% Unittest
\begin{frame}[fragile]
\frametitle{How to test code?}
\framesubtitle{Example python libraries}
\python
\lstset{basicstyle=\ttfamily\tiny}
\begin{exampleblock}{unittest}
A solid testing library that provides many functions to help you test units of code.
\lstinputlisting{code_examples/unittest_example.py}
\vfill
{\tiny\url{https://docs.python.org/2/library/unittest.html}}
\end{exampleblock}
\end{frame}

% Doctest
\begin{frame}[fragile]
\frametitle{How to test code?}
\framesubtitle{Example python libraries}
\python
\lstset{basicstyle=\ttfamily\tiny}
\begin{exampleblock}{doctest}
Write test in your documentation strings.
\lstinputlisting{code_examples/doctest_example.py}
\vfill
{\tiny\url{https://docs.python.org/2/library/doctest.html}}
\end{exampleblock}
\end{frame}


% Nosetests
\begin{frame}[fragile]
\frametitle{How to test code?}
\framesubtitle{Example python libraries}
\python
\begin{exampleblock}{nosetests}
Adds test discovery on top of available tools.

\lstinputlisting{code_examples/test_nose_example.py}
Try commands such as:
\begin{itemize}
  \item nosetests
  \item nosetests unittest\_example.py
  \item nosetests {-{}-}with-doctest
\end{itemize}
\vfill
{\tiny \url{https://nose.readthedocs.org/en/latest}}

\end{exampleblock}


\end{frame}

\section{Good testing practices}
\tocframe
%Good testing practice
\begin{frame}
  \frametitle{Good testing practices}
  \framesubtitle{Some things to keep in mind when writing tests.}
  \begin{block}{Attempt to at least:}
    \begin{itemize}
      \item<alert@2>  Focus on small units of functionality and ensure correctness
      \item<alert@3,7>  Tests must be able to run independently (see Mock for help)
      \item<alert@4>  Keep them fast! Slowness discourages running tests.
      \item<alert@5>  Write a test for debugging.
      \item<alert@6>  Keep your code clear, keep your tests clearer!
    \end{itemize}
    \vfill
      {\tiny Inspired by: The Hitchhiker’s Guide to Python}
  \end{block}

  {\tiny \url{https://github.com/kennethreitz/python-guide}}
\end{frame}

% Mock example
\begin{frame}[fragile]
  \frametitle{Mock, a useful tool in testing}
  \framesubtitle{Keeping your tests independent.}
  \python
  \lstset{basicstyle=\ttfamily\tiny}
  \begin{exampleblock}{Mock}
    Making sure that a method was called:
    \lstinputlisting{code_examples/mock_example.py}
  \end{exampleblock}
  \vfill
  {\tiny \url{http://www.voidspace.org.uk/python/mock/index.html}}
\end{frame}
\section{Alternate Testing Paradigms}
\subsection{Test-Driven Development}
\tocframe
% Different testing paradigm TDD
\begin{frame}
  \frametitle{Test-Driven Development}
  \framesubtitle{Developing code starting with tests}
  \begin{block}{How does it differ from conventional unit testing?}
    \begin{itemize}
      \item Instead of units of code, one first writes tests of the unit
      \item Then, you fill in the code to make the tests pass
      \item Finally, verify that the test succeeds.
      
    \end{itemize}

  \end{block}

\end{frame}

%uses/ advantages of tdd 
\begin{frame}
  \frametitle{Test-Driven Development}
  \framesubtitle{Developing code starting with tests}
  \begin{block}{What are the advantages?}
    \begin{itemize}
      \item Keeps code concise
      \item Clear purpose for implemented code
      \item \textbf{K}eep \textbf{I}t \textbf{S}imple \textbf{S}tupid!
      \item \textbf{Y}ou \textbf{A}ren't \textbf{G}onna \textbf{N}eed \textbf{I}t.
      
    \end{itemize}

  \end{block}

\end{frame}

\subsection{Behavior-Driven Development}
\tocframe
%use bdd
\begin{frame}
  \frametitle{Behavior-Driven Development}
  \framesubtitle{Developing from the end-users perspective.}
  \begin{block}{What are the key ideas?}
    \begin{itemize}
      \item<alert@2> The intended behavior of a program determines the development strategy
      \item<alert@3> From the start, you decide \emph{what} should be tested.
      \item<alert@4> Using object-oriented natural language to describe behavior
      \item<alert@5> Allow non-experts to design software and tests
      \item<alert@6> Then have programmers implement them
    \end{itemize}

  \end{block}

  
\end{frame}

\subsubsection{Cucumber and Behave}
%cucumber , behave
\begin{frame}
  \frametitle{Cucumber and Behave}
  \framesubtitle{Examples of Behavior-Driven Development tools}
  \begin{block}{What are they all about?}
    \begin{itemize}
      \item Cucumber (Ruby), Behave(Python), Specflow (.NET) et al.
      \item The behavior of the program is written down in a business readable, domain specific language called \emph{Gherkin}
      \item This is then parsed and matched to functions called \emph{step functions}
      \item The step functions are written in actual programming languages
      \item They test other underlying code and take input from the Gherkin parser
    \end{itemize}
  
  \end{block}
  \vfill
  {\tiny\url{https://github.com/cucumber/cucumber/wiki/Gherkin}}
\end{frame}

%gherkin
%example

\begin{frame}[fragile]
\frametitle{Gherkin syntax}
\gherkin
\begin{exampleblock}{A simple example}
\lstinputlisting{code_examples/features/example.feature}
\end{exampleblock}
\end{frame}

\begin{frame}[fragile]
\frametitle{Step methods}
\framesubtitle{The underlying python code}
\python
\begin{exampleblock}{A simple example}
\lstinputlisting{code_examples/features/steps/example_steps.py}
\end{exampleblock}
\end{frame}
  
%sources
\section*{Sources}
\begin{frame}
  \frametitle{Useful sources of information}
  \framesubtitle{..the wiki is not enough..}
  \begin{alertblock}{Urls}
    \begin{itemize}
      \item Hitchhiker's Guide to Python : \url{https://github.com/kennethreitz/python-guide}
      \item doctest, unittest: \url{https://docs.python.org/2/library/development.html}
      \item nose:  \url{https://nose.readthedocs.org}
      \item mock:  \url{http://www.voidspace.org.uk/python/mock/mock.html}
      \item behave:  \url{http://pythonhosted.org/behave/}
      \item cucumber: \url{http://cukes.info}
    \end{itemize}

  \end{alertblock}

\end{frame}


\end{document}
